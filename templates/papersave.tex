\documentclass{article}

\usepackage{fontspec}
\usepackage{calc}
\usepackage{geometry}
\usepackage{graphicx}

\geometry{
  a4paper,
  total={170mm,257mm},
  left=20mm,
  top=20mm,
}

\usepackage{fancyhdr}
\renewcommand{\headrulewidth}{0pt}
\usepackage{lastpage} 
\pagestyle{fancy}
\fancyhf{}
\fancyhead[L]{\small \filename{} -- \leftmark\rightmark}
\fancyhead[R]{\small Last modification: \lastmod}
\fancyfoot[L]{\small \filename{} -- \leftmark\rightmark}
\fancyfoot[C]{\thepage{} of \pageref{LastPage}\relax}
\fancyfoot[R]{\small generated with papersave}


\usepackage{xstring}
%% https://tex.stackexchange.com/questions/171007/split-a-character-string-n-by-n
%% Chunk HEX string in 8-chars sub strings
\def\split#1#2{%
  %% Split string into tempa and tempb 
  \StrSplit{#2}{#1}\tempa\tempb%
  %% Display tempa and empty it
  \tempa\let\tempa\empty%
  %
  \unless\ifx\tempb\empty%
  \def\tempa{%
    %% The string separator is an empty rule
    {\rule[0ex]{0.75em}{0pt}}%
    \split{#1}\tempb%
  }%
  \fi%
  \tempa%
}%

%\newfontfamily{\fixedw}[%
\usepackage{libertine}
\setmonofont[%
  Path = ./,
  UprightFont = *-light,
  ItalicFont = *-lightitalic,
  BoldFont = *-semibold,
  BoldItalicFont = *-semibolditalic,
  %% Mapping=mapping,
  %% Scale=0.90,
  %%FakeStretch=.88,
]{iosevka}



\def\filename{\code{\relax{}{{ .Filename }}\relax}}
\def\lastmod{\relax{}{{ .Fileinfo.ModTime }}\relax}

\newcommand{\code}[1]{\texttt{#1}}
\newcommand{\shasum}[1]{\code{#1}}
\newcommand{\tech}[1]{\textit{#1}}


\newcounter{linenumber}
\newcommand{\hashline}[2]{%
  \stepcounter{linenumber}%
  \arabic{linenumber} & \code{\split{8}{#1}} & \code{#2}\\
}



\begin{document}
\setlength{\parindent}{0cm}
\begin{center}
  {\Huge \code{\filename}\relax}%
  \vspace{2em}
  \begin{tabular}{ll}
    File size & {{ .Fileinfo.Size }} bytes (last modified \lastmod)\\
    File Sha256 & \shasum{\relax{}{{ .Checksum }}\relax}\\
    Base64 Sha256 & \shasum{\relax{}{{ .Base64Checksum }}\relax}\\
{{ if .Password }}
    GPG AES256 password & \verb|{{ .Password }}| \\
{{ end }}
  \end{tabular}
  \vspace{2em}
\end{center}

This is the paper backup of \filename. The original file has been compressed
using \tech{gzip} then {{ if .Password }}encrypted with a \tech{gpg}
compatible \tech{AES256} cipher and finally {{ end }}\tech{base64}
encoded. The \tech{base64} stream has been split into {{ .Blocks | len }}
blocks. For each block its QR-code and textual representations provided
along with the block stream \tech{sha256} sum (one line without any blank
character). The \tech{base64} encoded file can be restored by using any
QR-code scanner or OCR tool. Yet the content can be manually typed. To help
you each line can be verified using a \tech{IEEE-CRC32} checksum. As for the
block \tech{sha256} ckecksum, the whole \tech{base64} stream checksum has
been computed whitout any blank character.

\vspace{1em}

To compute a \tech{IEE-CRC32} sum:
\begin{center}
  \code{echo -n line | gzip | tail -c 8 | hexdump -n 4 -e '"0x\%.8x{\textbackslash}n"'}
\end{center}

To remove all blanks from the \tech{base64} stream you can run:
\begin{center}
  \code{cat \filename.b64 | sed 's/ *\#.*//;s/ +//;' | tr -d '{\textbackslash}n'}
\end{center}


The file restoration command (depending on your system the \code{-D} option
can be \code{-d}) from the \tech{base64} file is:
\begin{center}
  \code{cat \filename.b64 | base64 -D{{ if .Password }} | gpg -d --batch --passphrase '}\verb|{{ .Password }}|\code{'{{ end }} | zcat > \filename }
\end{center}

{{/*
The \tech{base64} data checksum has been computed on the \tech{base64}
1-line stream without any blank character. The original file \tech{sha256} 

To re
Restauration procedude:
\begin{itemize}
\item The orignial file has been split in {{ .Blocks | len }} blocks. Each
  QRCode and table represent one block. Regenerate the base64 file using
  either the QRcodes or an OCR tool or manually.
\item In case of manual typing each line can be checked with its IEEE-CRC32
  code. The whole sha256 is provided for each block.
\item IEEE-CRC32 can be checked using: \code{echo line | python3 -c 'import binascii;import sys;print("0x\%.8x" \% binascii.crc32(sys.stdin.buffer.read().strip()))'}
\item Check the sha256 of the whole base64 restored file. This checksum was
  computed as a one-line file without any blank character.
\item The base64 is the representation of the original file compressed with
  gzip.
\item To extract the original file: \code{ cat \filename.b64 | zcat > \filename }
\end{itemize}

*/}}
\vspace{1em}

{{ define "block-lines" }}
{{ block "block-qrcode" . }}{{ end }} 
{{ $llines := (len .Lines) | Add -1 }}
\vspace{0.5em}
  \begin{tabular}{rll}
{{/*    \multicolumn{3}{c}{Begin of block {{ .Number}}}\\ */}}
    \textbf{Line} & \multicolumn{1}{c}{\textbf{Block {{ .Number}} (lines {{ .LineStart }} to {{ .LineEnd }}) base64 data}\relax} & \textbf{CRC32} \\
{\rule[0ex]{0pt}{\normalbaselineskip}}
{{-   range $idx, $l := .Lines }}
{{ if (eq $idx $llines) }}
\rule[-0.55\normalbaselineskip]{0pt}{\normalbaselineskip}
{{ end }}
\hashline{%
  {{ $l.Data | String }}%
}{%
  {{ printf "0x%.8x" $l.CRC32 }}%
}%    
{{-    end }}
{{/* \multicolumn{3}{c}{End of block {{ .Number}}} \\ */}}
\multicolumn{3}{c}{\textbf{Sha256 of block {{ .Number}}:} \shasum{\relax{}{{ .Checksum }}\relax}}\\
  \end{tabular}
{{ end }}


{{ define "block-qrcode" }}
\frame{\includegraphics[height=7.5cm,keepaspectratio=true]{%
    {{ .QRCodeFile }}%
}}%
{{ end }}

\vspace{2em}

\begin{center}
  \textbf{\Large{\code{\filename}} starts here}
\end{center}

{{ $lblocks := len .Blocks }}
{{ range .Blocks | Range2 }}
{{-   $left := (index . 0) -}}
{{-   $right := (index . 1) -}}
\begin{center}

{{/*
  \begin{tabular}{cc}
    {{ block "block-qrcode" $left }}{{ end }} &
           {{-   if (ne 0 $right.Number) }}{{ block "block-qrcode" $right }}{{ end }}{{ end }} \\
    \textbf{Block {{ $left.Number }} } &     {{-   if (ne 0 $right.Number) }}\textbf{Block {{ $right.Number }} }{{ end }} \\
  \end{tabular}

  \vspace{1em}
*/}}
{{ block "block-lines" $left }}{{ end }}
{{-   if (ne 0 $right.Number) }}
\markboth{Block {{ $left.Number}}\relax}{\relax{ }--- Block {{ $right.Number}}\relax}
  \vspace{1em}
{{ block "block-lines" $right }}{{ end }}
{{ else }}
\markboth{Block {{ $left.Number}}\relax}{}
{{-   end }}

\end{center}
{{   if (not (or (eq $left.Number  $lblocks) (eq $right.Number $lblocks))) }}
\newpage
{{   end }}
{{ end }}


\begin{center}
  \textbf{\Large{\code{\filename} ends here}}
\end{center}
\end{document}
